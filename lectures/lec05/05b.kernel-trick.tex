%% LyX 2.3.4.2 created this file.  For more info, see http://www.lyx.org/.
%% Do not edit unless you really know what you are doing.
\documentclass[english,dvipsnames,aspectratio=169]{beamer}
\usepackage{mathptmx}
\usepackage{eulervm}
\usepackage[T1]{fontenc}
\usepackage[latin9]{inputenc}
\usepackage{babel}
\usepackage{amstext}
\usepackage{amssymb}
\usepackage{graphicx}
\usepackage{ifthen}
\usepackage{xcolor}
\usepackage{xspace}
\usepackage{tikz}
\usetikzlibrary{tikzmark}
\usetikzlibrary{calc}
\usepackage{pgfplots}
%\pgfplotsset{compat=1.17}
\usepackage{booktabs}
\usepackage{xpatch}

\usepackage{pgfpages}
\setbeamertemplate{note page}{\pagecolor{yellow!5}\insertnote}
%\setbeameroption{hide notes} % Only slides
%\setbeameroption{show only notes} % Only notes
%\setbeameroption{show notes on second screen=right} % Both

\xpatchcmd{\itemize}
  {\def\makelabel}
  {\ifnum\@itemdepth=1\relax
     \setlength\itemsep{2ex}% separation for first level
   \else
     \ifnum\@itemdepth=2\relax
       \setlength\itemsep{1ex}% separation for second level
     \else
       \ifnum\@itemdepth=3\relax
         \setlength\itemsep{0.5ex}% separation for third level
   \fi\fi\fi\def\makelabel
  }
 {}
 {}

\ifx\hypersetup\undefined
  \AtBeginDocument{%
    \hypersetup{unicode=true,pdfusetitle,
 bookmarks=true,bookmarksnumbered=false,bookmarksopen=false,
 breaklinks=false,pdfborder={0 0 0},pdfborderstyle={},backref=false,colorlinks=true,
 allcolors=NYUPurple,urlcolor=LightPurple}
  }
\else
  \hypersetup{unicode=true,pdfusetitle,
 bookmarks=true,bookmarksnumbered=false,bookmarksopen=false,
 breaklinks=false,pdfborder={0 0 0},pdfborderstyle={},backref=false,colorlinks=true,
 allcolors=NYUPurple,urlcolor=LightPurple}
\fi

\makeatletter

%%%%%%%%%%%%%%%%%%%%%%%%%%%%%% LyX specific LaTeX commands.
%% Because html converters don't know tabularnewline
\providecommand{\tabularnewline}{\\}

%%%%%%%%%%%%%%%%%%%%%%%%%%%%%% Textclass specific LaTeX commands.
% this default might be overridden by plain title style
\newcommand\makebeamertitle{\frame{\maketitle}}%
% (ERT) argument for the TOC
\AtBeginDocument{%
  \let\origtableofcontents=\tableofcontents
  \def\tableofcontents{\@ifnextchar[{\origtableofcontents}{\gobbletableofcontents}}
  \def\gobbletableofcontents#1{\origtableofcontents}
}

%%%%%%%%%%%%%%%%%%%%%%%%%%%%%% User specified LaTeX commands.
\usetheme{CambridgeUS} 
\beamertemplatenavigationsymbolsempty


% Set Color ==============================
\definecolor{NYUPurple}{RGB}{87,6,140}
\definecolor{LightPurple}{RGB}{165,11,255}


\setbeamercolor{title}{fg=NYUPurple}
\setbeamercolor{frametitle}{fg=NYUPurple}

\setbeamercolor{background canvas}{fg=NYUPurple, bg=white}
\setbeamercolor{background}{fg=black, bg=NYUPurple}

\setbeamercolor{palette primary}{fg=black, bg=gray!30!white}
\setbeamercolor{palette secondary}{fg=black, bg=gray!20!white}
\setbeamercolor{palette tertiary}{fg=gray!20!white, bg=NYUPurple}

\setbeamertemplate{headline}{}
\setbeamerfont{itemize/enumerate body}{}
\setbeamerfont{itemize/enumerate subbody}{size=\normalsize}
\setbeamerfont{itemize/enumerate subsubbody}{size=\normalsize}

\setbeamercolor{parttitle}{fg=NYUPurple}
\setbeamercolor{sectiontitle}{fg=NYUPurple}
\setbeamercolor{sectionname}{fg=NYUPurple}
\setbeamercolor{section page}{fg=NYUPurple}
%\setbeamercolor{description item}{fg=NYUPurple}
%\setbeamercolor{block title}{fg=NYUPurple}

\setbeamertemplate{blocks}[rounded][shadow=false]
\setbeamercolor{block body}{bg=normal text.bg!90!NYUPurple}
\setbeamercolor{block title}{bg=NYUPurple!30, fg=NYUPurple}



\AtBeginSection[]{
  \begin{frame}
  \vfill
  \centering
\setbeamercolor{section title}{fg=NYUPurple}
 \begin{beamercolorbox}[sep=8pt,center,shadow=true,rounded=true]{title}
    \usebeamerfont{title}\usebeamercolor[fg]{title}\insertsectionhead\par%
  \end{beamercolorbox}
  \vfill
  \end{frame}
}

\makeatother

\setlength{\parskip}{\medskipamount} 

\input ../macros

\begin{document}
\input ../rosenberg-macros

\title[DS-GA 1003]{Kernel Trick}
\author{He He}
\date{March 2, 2021}
\institute{CDS, NYU}

\makebeamertitle
\mode<article>{Just in article version}
\begin{frame}{SVM with Explicit Feature Map}
\begin{itemize}
\item Let $\psi:\cx\to\reals^{d}$ be a feature map.

\item The SVM objective (with explicit feature map):
\[
\min_{w\in\reals^{d}}\frac{1}{2}||w||^{2}+\frac{c}{n}\sum_{i=1}^{n}\max\left(0,1-y_{i}w^{T}\psi(x_{i})\right).
\]

\item Computation is costly if $d$ is large (e.g. with high-degree monomials) 
\item Last time we mentioned an equivalent optimization problem from Lagrangian
duality.
\end{itemize}
\end{frame}
%
\begin{frame}{SVM Dual Problem}
\begin{itemize}
\item By Lagrangian duality, it is equivalent to solve the following dual problem:
\end{itemize}
\begin{eqnarray*}
    \text{maximize} &  & \sum_{i=1}^{n}\alpha_{i}-\frac{1}{2}\sum_{i,j=1}^{n}\alpha_{i}\alpha_{j}y_{i}y_{j}
    {\color{blue}\psi\left(x_{j}\right)^{T}\psi(x_{i})}\\
\mbox{s.t.} &  & \sum_{i=1}^{n}\alpha_{i}y_{i}=0\qquad\text{and}\qquad\alpha_{i}\in\left[0,\frac{c}{n}\right]\quad\forall i.
\end{eqnarray*}

\begin{itemize}
\item If $\alpha^{*}$ is an optimal value, then
\[
w^{*}=\sum_{i=1}^{n}\alpha_{i}^{*}y_{i}\psi(x_{i})\qquad\text{and}\qquad\hat{f}(x)=\sum_{i=1}^{n}\alpha_{i}^{*}y_{i}
        {\color{blue}\psi(x_{i})^{T}\psi(x)}.
\]
\end{itemize}

\begin{itemize}
    \item \head{Key observation}: $\psi\left(x\right)$ only shows up in \hl{inner products} with
        another $\psi(x')$ \emph{for both training and inference}.
\end{itemize}
\end{frame}
%
\begin{frame}
    {Compute the Inner Products}
    Consider 2D data. Let's introduce \hl{degree-2 monomials} using $\psi:\reals^2\rightarrow\reals^3$.
    $$
    (x_1, x_2) \mapsto (x_1^2, \sqrt{2}x_1x_2, x_2^2) .
    $$
    The inner product is
    \begin{align*}
        \psi(x)^T\psi(x') &= x_{1}^2 {x'_{1}}^2 +
        (\sqrt{2}x_{1}x_{2})(\sqrt{2}x'_{1}x'_{2}) +
        x_{2}^2 {x'_{2}}^2 \\
        &= (x_{1} x'_{1})^2 + 2(x_{1} x'_{1})(x_{2} x'_{2}) + (x_{2} x'_{2})^2 \\
        &= (x_{1}x'_{1} + x_{2}x'_{2})^2 \\
        &= (x^Tx')^2
    \end{align*}
    We can calculate the inner product $\psi(x)^T\psi(x')$ without accessing the features $\psi(x)$!
\end{frame}
%
\begin{frame}
    {Compute the Inner Products}
    Now, consider \hl{monomials up to degree-2}:
    $$
    (x_1, x_2) \mapsto (1, \sqrt{2}x_1, \sqrt{2}x_2, x_1^2, \sqrt{2}x_1x_2, x_2^2) .
    $$
    The inner product can be computed by
    $$
    \psi(x)^T\psi(x') = (1 + x^Tx')^2 \quad \text{(check)}.
    $$
    More generally, for features maps producing monomials up to degree-$p$, we have
    $$
    \psi(x)^T\psi(x') = (1 + x^Tx')^p . 
    $$
    (Note that the coefficients of each monomial in $\psi$ may not be 1)

    \textbf{Kernel trick}: we do not need explicit features to calculate inner products.
    \begin{itemize}
        \setlength\itemsep{1ex}
        \item Using explicit features: $O(d^p)$
        \item Using implicit computation: $O(d)$
    \end{itemize}
\end{frame}
%
\section{Kernel Function}
\begin{frame}{The Kernel Function}
\begin{itemize}
    \setlength\itemsep{1ex}
\item \textbf{Input space}: $\cx$

\item \textbf{Feature space}: $\ch$ (a Hilbert space, e.g. $\reals^{d}$)

\item \textbf{Feature map}: $\psi:\cx\to\ch$

\item The \textbf{kernel function} corresponding to $\psi$ is 
\[
k(x,x')=\left\langle \psi(x),\psi(x')\right\rangle ,
\]
where $\left\langle \cdot,\cdot\right\rangle $ is the inner product
associated with $\ch$.
\end{itemize}

Why introduce this new notation $k(x,x')$?
\begin{itemize}
    \item We can often evaluate $k(x,x')$ without explicitly computing $\psi(x)$ and $\psi(x')$.
\end{itemize}
When can we use the kernel trick?
\end{frame}
%
\begin{frame}{Some Methods Can Be ``Kernelized''}
\begin{definition}
A method is \textbf{kernelized }if every feature vector $\psi(x)$
only appears inside an inner product with another feature vector $\psi(x')$.
This applies to both the optimization problem and the prediction function.
\end{definition}

The SVM Dual is a kernelization of the original SVM formulation. 

Optimization:
    \vspace{-1ex}
\begin{eqnarray*}
    \text{maximize} &  & \sum_{i=1}^{n}\alpha_{i}-\frac{1}{2}\sum_{i,j=1}^{n}\alpha_{i}\alpha_{j}y_{i}y_{j}
    {\color{blue}\psi\left(x_{j}\right)^{T}\psi(x_{i})}\\
\mbox{s.t.} &  & \sum_{i=1}^{n}\alpha_{i}y_{i}=0\qquad\text{and}\qquad\alpha_{i}\in\left[0,\frac{c}{n}\right]\quad\forall i.
\end{eqnarray*}

Prediction:
    \vspace{-1ex}
\[
\hat{f}(x)=\sum_{i=1}^{n}\alpha_{i}^{*}y_{i}
        {\color{blue}\psi(x_{i})^{T}\psi(x)}.
\]

\end{frame}
%
\begin{frame}{The Kernel Matrix}
\begin{definition}
The \textbf{kernel matrix} for a kernel $k$ on $x_{1},\ldots,x_{n}\in\cx$
is
\[
K=\begin{pmatrix}k(x_{i},x_{j})\end{pmatrix}_{i,j}=\begin{pmatrix}k(x_{1},x_{1}) & \cdots & k(x_{1},x_{n})\\
\vdots & \ddots & \cdots\\
k(x_{n},x_{1}) & \cdots & k(x_{n},x_{n})
\end{pmatrix}\in\reals^{n\times n}.
\]
\end{definition}

\begin{itemize}
\item In ML this is also called a \textbf{Gram matrix}, but traditionally
(in linear algebra),
Gram matrices are defined without reference to a kernel or feature
map.
\end{itemize}
\end{frame}
%
\begin{frame}{The Kernel Matrix}
\begin{itemize}
    \item The kernel matrix \hl{summarizes all the information} we need about the
training inputs $x_{1},\ldots,x_{n}$ to solve a kernelized optimization
problem.
\item In the kernelized SVM, we can replace $\psi(x_{i})^{T}\psi(x_{j})$
with $K_{ij}$:
\begin{eqnarray*}
    \text{maximize}_{\alpha} &  & \sum_{i=1}^{n}\alpha_{i}-\frac{1}{2}\sum_{i,j=1}^{n}\alpha_{i}\alpha_{j}y_{i}y_{j}{\color{blue}K_{ij}}\\
\mbox{s.t.} &  & \sum_{i=1}^{n}\alpha_{i}y_{i}=0\qquad\text{and}\qquad\alpha_{i}\in\left[0,\frac{c}{n}\right]\;i=1,\ldots,n.
\end{eqnarray*}
\end{itemize}
\end{frame}
%
\begin{frame}{Kernel Methods}
Given a kernelized ML algorithm (i.e. all $\psi(x)$'s show up as
$\left\langle \psi(x),\psi(x')\right\rangle $),

\begin{itemize}

\item Can swap out the inner product for a new kernel function.

\item New kernel may correspond to a \al{very high-dimensional} feature space.

\item Once the kernel matrix is computed, the computational cost \hl{depends
    on number of data points} $n$, rather than the dimension of feature space $d$.

\item Useful when $d >> n$.

\item Computing the kernel matrix may still depend on $d$
    and the essence of the \textbf{trick} is getting around this $O(d)$ dependence.
\end{itemize}
\end{frame}
%
\section{Example Kernels}
\begin{frame}{Kernels as Similarity Scores}
\begin{itemize}
    \item Often useful to think of the $k(x, x')$ as a \textbf{similarity
score} for $x$ and $x'$.
\item We can design similarity functions without thinking about the explicit feature map, e.g. ``string kernels'', ``graph kerners''.
\item How do we know that our kernel functions actually correspond to inner products in
    some feature space?
\end{itemize}
\end{frame}
%
\begin{frame}{How to Get Kernels?}
\begin{itemize}
    \item Explicitly construct $\psi(x):\cx\to\reals^{d}$ (e.g. monomials) and define $k(x,x')=\psi(x)^{T}\psi(x')$.

\item Directly define the kernel function $k(x,x')$ (``similarity score''), and \hl{verify it corresponds
    to $\left\langle \psi(x),\psi(x')\right\rangle $ for some $\psi$}.
\end{itemize}

There are many theorems to help us with the second approach.

\end{frame}

\begin{frame}{Linear Algebra Review: Positive Semidefinite Matrices}
\begin{definition}
A real, symmetric matrix $M\in\reals^{n\times n}$ is \textbf{positive
semidefinite (psd)} if for any $x\in\reals^{n}$, 
\[
x^{T}Mx\ge0.
\]

\end{definition}

\begin{theorem}
The following conditions are each necessary and sufficient for a symmetric
matrix $M$ to be positive semidefinite:
\begin{itemize}
\item $M$ can be factorized as $M=R^{T}R$, for some matrix $R$.
\item All eigenvalues of $M$ are greater than or equal to $0$.
\end{itemize}
\end{theorem}
\end{frame}
%
\begin{frame}{Positive Definite Kernel}
\begin{definition}
A symmetric function $k:\cx\times\cx\to\reals$ is a \textbf{positive
    definite (pd)} kernel on $\cx$ if for any finite set $\left\{ x_{1},\ldots,x_{n}\right\} \in\cx$ ($n\in\BN$),
the kernel matrix on this set 
\[
K=\begin{pmatrix}k(x_{i},x_{j})\end{pmatrix}_{i,j}=\begin{pmatrix}k(x_{1},x_{1}) & \cdots & k(x_{1},x_{n})\\
\vdots & \ddots & \cdots\\
k(x_{n},x_{1}) & \cdots & k(x_{n},x_{n})
\end{pmatrix}
\]
is a positive semidefinite matrix.
\end{definition}

\begin{itemize}
    \item Symmetric: $k(x, x') = k(x', x)$
    \item The kernel matrix needs to be positive semidefinite for \hl{any} finite set of points.
    \item Equivalent definition: $\sum_{i=1}^n\sum_{j=1}^n \alpha_i\alpha_j k(x_i, x_j) \ge 0$ given $\alpha_i\in\reals\;\forall i$.
\end{itemize}
\end{frame}
%
\begin{frame}{Mercer's Theorem}
\begin{theorem}
A symmetric function $k(x,x')$ can be expressed as an inner product
\[
k(x,x')=\left\langle \psi(x),\psi(x')\right\rangle 
\]
for some $\psi$ if and only if $k(x,x')$ is \textbf{positive definite}.
\end{theorem}

    \begin{itemize}
        \item Proving a kernel function is positive definite is typically not easy.
        \item But we can construct new kernels from valid kernels.
    \end{itemize}
\end{frame}
%
\begin{frame}{Generating New Kernels from Old}
\begin{itemize}
\item Suppose $k,k_{1},k_{2}:\cx\times\cx\to\reals$ are pd kernels. Then
so are the following:
\begin{eqnarray*}
    k_{\mbox{new}}(x,x') & = & \alpha k(x,x') \quad \text{for } \alpha \ge 0 \quad \text{(non-negative scaling)}\\
    k_{\mbox{new}}(x,x') & = & k_{1}(x,x')+k_{2}(x,x') \quad\text{(sum)}\\
    k_{\mbox{new}}(x,x') & = & k_{1}(x,x')k_{2}(x,x') \quad\text{(product)}\\
    k_{\mbox{new}}(x,x') & = & k(\psi(x), \psi(x'))\mbox{ for any function \ensuremath{\psi(\cdot)}} \quad\text{(recursion)}\\
    k_{\mbox{new}}(x,x') & = & f(x)f(x')\mbox{ for any function \ensuremath{f(\cdot)}} \quad\text{($f$ as 1D feature map)}\\
\end{eqnarray*}
\item Lots more theorems to help you construct new kernels from old.
\end{itemize}
    \let\thefootnote\relax\footnotetext{\tiny{Based on Mark Schmidt's slides:\url{https://www.cs.ubc.ca/~schmidtm/Courses/540-W19/L12.5.pdf}}}
\end{frame}

\begin{frame}{Linear Kernel}
 
\begin{itemize}
\item Input space: $\cx=\reals^{d}$
\item Feature space: $\ch=\reals^{d}$, with standard inner product

\item Feature map
\[
\psi(x)=x
\]
\item Kernel: 
\[
k(x,x')=x^{T}x'
\]
\end{itemize}
\end{frame}
%
\begin{frame}{Quadratic Kernel in $\reals^{d}$}
 
\begin{itemize}
\item Input space $\cx=\reals^{d}$
\item Feature space: $\ch=\reals^{D}$, where $D=d+{d \choose 2}\approx d^{2}/2$.
\item Feature map:
\[
\ensuremath{\psi(x)=(x_{1},\ldots,x_{d},x_{1}^{2},\ldots,x_{d}^{2},\sqrt{2}x_{1}x_{2},\ldots,\sqrt{2}x_{i}x_{j},\ldots\sqrt{2}x_{d-1}x_{d})^{T}}
\]


\item Then for $\forall x,x'\in\reals^{d}$
\begin{eqnarray*}
k(x,x') & = & \left\langle \psi(x),\psi(x')\right\rangle \\
& = & \left\langle x,x'\right\rangle +\left\langle x,x'\right\rangle ^{2}
\end{eqnarray*}


\item Computation for inner product with explicit mapping: $O(d^{2})$
\item Computation for implicit kernel calculation: $O(d)$.
\end{itemize}
\end{frame}
%
\begin{frame}{Polynomial Kernel in $\reals^{d}$}
 
\begin{itemize}
\item Input space $\cx=\reals^{d}$
\item Kernel function:
\[
k(x,x')=\left(1+\left\langle x,x'\right\rangle \right)^{M}
\]


\item Corresponds to a feature map with all monomials up to degree $M$.

\item For any $M$, computing the kernel has same computational cost

\item Cost of explicit inner product computation grows rapidly in $M$.
\end{itemize}
\end{frame}

\begin{frame}{Radial Basis Function (RBF) / Gaussian Kernel}
Input space $\cx=\reals^{d}$
\[
k(x,x')=\exp\left(-\frac{\|x-x'\|^{2}}{2\sigma^{2}}\right),
\]
where $\sigma^{2}$ is known as the bandwidth parameter.
 
\begin{itemize}

\item Probably the most common nonlinear kernel.

\item Does it act like a similarity score?

\item Why ``radial''?

\item Have we departed from our ``inner product of feature vector'' recipe?
    \begin{itemize}
\item Yes and no: corresponds to an infinite dimensional feature vector
    \end{itemize}
\end{itemize}
\end{frame}
%
\begin{frame}
    {Remaining Questions}
    Our current recipe:\\
    \begin{itemize}
        \item Recognize kernelized problem: $\psi(x)$ only occur in inner products $\psi(x)^T\psi(x')$
        \item Pick a kernel function (``similarity score'')
        \item Compute the kernel matrix ($n$ by $n$ where $n$ is the dataset size)
        \item Optimize the model and make predictions by accessing the kernel matrix
    \end{itemize}
    Next: When can we apply kernelization?
\end{frame}

\end{document}
